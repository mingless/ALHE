\chapter{Treść projektu}
\section{Zmodyfikowany algorytm ewolucji różnicowej}
\textit{W ramach projektu należy przygotować implementację zmodyfikowanego algorytmu ewolucji różnicowej korzystającego z zapisanej historii populacji oraz punktu środkowego (metoda DES --- dokładny opis w artykule). Kod trzeba przygotować w formie biblioteki, a przeprowadzić jego testy na wybranym zestawie zadań optymalizacji. Za wybranie projektu można otrzymać dodatkowe 5 pkt.}

\chapter{Dokumentacja końcowa}
\section{Wyjaśnienie algorytmu}
Algorytm DES w uproszczeniu można uznać za pośredni do CMA-ES oraz ewolucji różnicowej. Podobnie jak CMA-ES algorytm ten przemieszcza populację w kierunku wyznaczonym ze średniej populacji do średniej pewnej grupy najlepszych osobników, wykorzystuje też różnicę między wylosowanymi osobnikami z populacji podobnie do ewolucji różnicowej.

W skrócie algorytm można opisać w następujących punktach:

\begin{enumerate}
    \item Inicjalizacja parametrów i populacji początkowej.
    \item Obliczenie średniej z populacji.
    \item Sortowanie osobników populacji zgodnie z funkcją celu.
    \item Obliczenie średniej z $\mu$ najlepszych osobników.
    \item Aktualizacja każdego z osobników w oparciu o obliczone średnie, parametry \{$c$, $F$\}, pewne zmienne losowe o danych rozkładach oraz zadany szum.
    \item W wypadku nieosiągnięcia warunku stopu powrót do punktu drugiego.
\end{enumerate}

W projekcie nie zaimplementowano stosowania ograniczeń. Można je za to uwzględnić przy stosowaniu naszej funkcji odpowiednio modyfikując funkcję celu - tj. dodać do niej pewną karę przy przekroczeniu granic.

\section{Implementacja algorytmu}
Zgodnie z założeniem algorytm zaimplementowany został w języku R i dostępny jest z poziomu zbudowanego pakietu. Elementy pamięciowe algorytmu przechowywane są w listach, pozwala to na prostą i intuicyjną implementację, aczkolwiek możliwe, że wybranie list zamiast macierzy negatywnie wpłynęło na wydajność implementacji. Końcową implementację można zobaczyć w pliku źródłowym \verb|des.R| zamieszczonym w przesłanej bibliotece. Są tam zapisane dwie funkcje - przykładową funkcję inicjalizacji populacji początkowej \verb|desUniformInit| i właściwego algorytmu \verb|des|. Szczegóły przyjmowanych parametrów zostały zapisane w dokumentacji biblioteki, którą można obejrzeć za pomocą komendy \verb|?<funkcja>|.

W implementacji algorytmu zezwolono na ręczne ustawienie następujących parametrów, oprócz tych definiujących samą funkcję celu:
\begin{itemize}
\item[--] rozmiar całkowitej populacji $ \lambda $
\item[--] rozmiar potomstwa/elity $\mu$
\item[--] współczynnik skalowania $F$
\item[--] współczynnik migracji punktu środkowego populacji $c$
\item[--] długość historii populacji $H$
\item[--] natężenie szumu $\epsilon$
\item[--] zmiana funkcji generującej populację początkową
\item[--] wstępna przestrzeń przeszukiwania stosowana przy domyślnej generacji początkowej populacji
\item[--] warunki stopu - maksymalna ilość iteracji i minimalna dopuszczalna wartość funkcji celu
\end{itemize}

Większość z podanych algorytmów ustalono na wartości domyślne pobrane z podanego w poleceniu artykułu - tj. wartości stosowanych w połączeniu z funkcjami benchmarkowymi CEC2017.

\section{Działanie algorytmu}
Działanie algorytmu można zaobserwować na poniższych wykresach, na funkcjach dwuargumentowych celem łatwiejszej reprezentacji wyników.

Na rysunku \ref{R1} przedstawiono przykładową optymalizację funkcji michalewicza. Można zobaczyć ją samemu uruchamiając zamieszczone w bibliotece demo \verb|simple_demo.R| - można ujrzeć wtedy na własnym  wykresie wynik działania funkcji i ewentualnie poeksperymentować z parametrami algorytmu.

\begin{figure}
\includegraphics{./example.pdf}
\caption{Optymalizacja funkcji michalewicza za pomocą DES. Stosowane niedomyślne parametry: $ H = 12 $, $ \epsilon = \num{1e-1} $, $ c = 0.6 $, iteracje$ =100 $}
\label{R1}
\end{figure}

Uruchomienie tego dema nie skutkuje w każdym przypadku zbiegnięciem się do zera globalnego znajdującego się w punkcie $ (2.20,1.57) $ - wynika to z połączenia małego stosowanego współczynnika błędu, wysokiej losowości rozmieszczenia wstępnej populacji, a także bardzo małej ilości iteracji. Można jednak zauważyć, że generalna populacja po wpadnięciu w jedną z dolin zaczyna ją eksplorować aż do znalezienia większego minimum, którego okolicę wtedy zaczyna eksploatować.

Inne przykłady przedstawiono na rysunkach \ref{R2} i \ref{R3}. Na obu z nich widać optymalizację funkcji levy'ego z inicjalizacją w małym kwadracie $ x \in (-10,-9), y\in (-10,-9) $. Służy to zaprezentowaniu możliwości utykania w lokalnym minimum przy niefortunnym usytuowaniu początkowym (czemu częściowo zapobiega zastosowanie większej wartości parametru $ H $, powiększającym pulę osobników używanych do reprodukcji). Na rysunku \ref{R2}, przy zastosowaniu $ \epsilon = \num{1e-1} $ funkcja utyka w minimum lokalnym. Na \ref{R3}, przy większym szumie $ \epsilon = \num{1} $ funkcja jest w stanie w zniego wyjść i zacząć eksploatować właściwe minimum funkcji.
 
\begin{figure}
	\includegraphics{./lownoise.pdf}
	\caption{Optymalizacja funkcji levy'ego za pomocą DES. Stosowane niedomyślne parametry: $ \epsilon = \num{1e-1} $, $ c = 0.9 $, iteracje$ =4000 $}
	\label{R2}
\end{figure}


\begin{figure}
	\includegraphics{./highnoise.pdf}
	\caption{Optymalizacja funkcji levy'ego za pomocą DES. Stosowane niedomyślne parametry: $ \epsilon = \num{1e0} $, $ c = 0.9 $, iteracje = $4000 $}
	\label{R3}
\end{figure}

\section{Przeprowadzone testy na zbiorze CEC2017}


Pierwotnie planowane było przeprowadzenie strojenia parametrów algorytmu DES na zbiorze testowym, niestety skończona implementacja obudowana algorytmem GA w celu optymalizacji (z kilkoma parametrami całkowitoliczbowymi) potrzebowała znacznie zbyt dużo czasu do wykonania pojedyncze uruchomienie DES na problemie z CEC2017 z zastosowaniem \num{100000} iteracji wymagało około 20 minut. Uwzględniając konieczność wykonania co najmniej 10--20 uruchomień algortymu DES w pojedynczej iteracji algorytmu GA oraz konieczność przeprowadzenia optymalizacji na więcej niż jednej funkcji otrzymaliśmy przewidywany czas obliczeń znacznie wykraczający poza nasze możliwości. Po przeprowadzeniu dalszych testów ustaliliśmy, że w zakresie naszych możliwości będzie sprawdzenie algorytmu z parametrami zadanymi w artykule, na zbiorze 5 funkcji 10-wymiarowych w 10 iteracjach.

Poniżej zamieszczono wyniki statystyk z 10 iteracji dla kilku losowo wybranych funkcji z zestawu CEC2017. Tak jak zaproponowano w definicji konkursu CEC, wartości poniżej rzędu e-09 lub mniejszego zostały zastąpione zerami.

\begin{table}[ht]
\caption{Wyniki dla funkcji 10-wymiarowych}
\label{t1}
\centering
\sisetup{table-format = 1.4e-1}
\begin{small}
    \begin{tabular}{|c|c|c|c|c|c|}
        \hline
        \multicolumn{1}{|c|}{Nr funkcji\rule{0pt}{3.5mm}} & Najlepszy & Najgorszy & Mediana & Średnia & Odchylenie \\ \hline
        2\rule{0pt}{3.5mm}    &       0  &   1e+00    & 0   & 0    & 4.47e-01  \\
        4                     &       0  &   0    & 0   & 0    & 0   \\
        7                     &       1.32e+01  &   1.62e+01    & 1.43e+01   & 1.45e+01    & 1.15e+00   \\
        17                    &       2.29e+01  &   2.93e+01    & 2.54e+01   & 2.57e+01    & 2.62e+00   \\
        23                    &       3.04e+02  &   3.11e+02    & 3.09e+02   & 3.08e+02    & 2.70e+00   \\ \hline
    \end{tabular}
\end{small}
\end{table}

Jak widać wyniki uzyskane w testach mają rzędy wielkości porównywalne z wynikami uzyskanymi w artykule.

